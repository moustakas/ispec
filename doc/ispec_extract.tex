\documentclass[12pt,preprint]{aastex}

\input{/home/ioannis/bin/latex/myprefs}
\newcommand{\rk}{{\tt RKSPEC}}
\newcommand{\idl}{{\tt IDL}}

\begin{document}
\pagestyle{plain}

\title{Extracting Spectra with \rk}
\author{J. Moustakas\altaffilmark{1}}
%\date{August 2001}
\affil{Steward Observatory, University of Arizona, Tucson, AZ 85721}
\affil{2003 January 14}
\altaffiltext{1}{{\tt jmoustakas@as.arizona.edu}}

We assume you have a two-dimensional spectrum called
\emph{galaxy.fits} that has been fully reduced with \rk.  To extract a
spectrum simply call the routine RKSPEC, which is a wrapper for
RKEXTRACT, the spectral extraction workhorse.

\begin{verbatim}
       IDL> rkspec, 'galaxy.fits', info
\end{verbatim}

\noindent Two windows will appear.  The window on the right shows the
spatial profile of the object, the aperture window, the sky windows,
and information from the header like the airmass.  The window on the
left shows the extracted one-dimensional spectrum.  The output
\emph{info} is a data structure that shows all the extraction
parameters, one-dimensional spectra, the fitted sky image (SKYFIT),
and the sky-subtracted two-dimensional image (IMNOSKY), among other
things:

\begin{verbatim}
       IDL> help, info, /str
       ** Structure <81d0394>, 20 tags, length=1301408, data length=1301406, refs=1:
          SPEC            FLOAT     Array[1194]
          WAVE            FLOAT     Array[1194]
          SIGSPEC         FLOAT     Array[1194]
          SKY             FLOAT     Array[1194]
          MASK            BYTE      Array[1194]
          HEADER          STRING    Array[92]
          REFROW          LONG                63
          REFCOL          LONG               596
          REFWAVE         FLOAT           5270.38
          APERTURE        FLOAT           20.0000
          CENTER          DOUBLE    Array[1194]
          LOWER           DOUBLE    Array[1194]
          UPPER           DOUBLE    Array[1194]
          SKYAP           LONG      Array[2]
          SKYLOW          LONG      Array[1194]
          SKYUP           LONG      Array[1194]
          FLANKING        FLOAT     Array[2]
          SKYMETHOD       LONG                 3
          SKYFIT          FLOAT     Array[1194, 130]
          IMNOSKY         FLOAT     Array[1194, 130]
\end{verbatim}

\noindent The one-dimensional spectrum can be written to a FITS file
with the WFITS keyword:

\begin{verbatim}
       IDL> rkspec, 'galaxy.fits', /wfits
\end{verbatim}

\noindent The output file name will be \emph{galaxy\_020.ms.fits}.  The
``020'' specifies the extraction aperture in arcseconds and the ``ms''
indicates that the file is a multi-extension FITS file.  To read or
plot the one-dimensional spectrum try

\begin{verbatim}
       IDL> spec = rd1dspec('galaxy_020.ms.fits')
       IDL> help, spec, /str
       ** Strucature <81dbc54>, 10 tags, length=25024, data length=25024, refs=1:
       	  SPECNAME        STRING    'galaxy_020.ms.fits'
       	  DATAPATH        STRING    '/home/ioannis/'
       	  OBJECT          STRING    'N1377 nuc'
       	  HEADER          STRING    Array[92]
       	  NPIX            LONG              1194
       	  SPEC            FLOAT     Array[1194]
       	  SIGSPEC         FLOAT     Array[1194]
       	  SKY             FLOAT     Array[1194]
       	  MASK            FLOAT     Array[1194]
       	  WAVE            FLOAT     Array[1194]
       IDL> plot1dspec, 'galaxy_020.ms.fits'
\end{verbatim}

\noindent The extracted spectrum has the spectrum, the error spectrum,
the sky spectrum, and the bad pixel spectrum in each of four FITS
extensions.  For example, to plot the signal-to-noise spectrum try

\begin{verbatim}
       IDL> plot, spec.wave, spec.spec/spec.sigspec, ps=10, xsty=3, ysty=3
\end{verbatim}

\noindent All the extraction parameters are documented in RKEXTRACT.
Here is an example.  Extract and write out a $5\arcsec$ spectrum with
second-order tracing and $60\arcsec$ sky apertures.

\begin{verbatim}
       IDL> rkspec, 'galaxy.fits', info, aperture=5, skyaperture=60, $
       IDL>   /tracespec, traceorder=2, /wfits
\end{verbatim}

\end{document}