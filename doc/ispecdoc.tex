\documentclass[12pt,preprint]{aastex}

\def\li{\item[$\bullet$]}
\def\bli{\begin{enumerate}}
\def\eli{\end{enumerate}}

\begin{document}
\pagestyle{plain}

\title{RKSPEC: A Data Reduction Package for the B\&C Spectrograph}
\author{J. Moustakas\altaffilmark{1}}
%\date{August 2001}
\affil{Steward Observatory, University of Arizona, Tucson, AZ 85721}
\altaffiltext{1}{{\tt jmoustakas@as.arizona.edu}}

\section{Recipe}

Run IHEADER_CHECK and make sure all the right keywords are where they
need to be.  Make file lists with ISPEC_MAKELISTS, /all




\section{Introductory Comments}\label{sec:intro}

{\tt iSPEC} is an improvement over the existing IRAF\footnote{IRAF is
distributed by the National Optical Astronomy Observatories, which are
operated by the Association of Universities for Research in Astronomy,
Inc., under cooperative agreement with the National Science
Foundation.} spectroscopy packages in several important ways.
Clarity, flexibility, and speed are at the center of the philosophy
behind {\tt iSPEC}.  Every manipulation of the data in the reduction
process is accessible to the user via well-defined input parameters
and can be critically assessed with practical quality-assurance plots.
Alternatively, once input parameter files have been created, the data
can be reduced with batch scripts with no user interaction, including
automated wavelength calibration, cosmic-ray rejection, and
flux-calibration.  Two-dimensional error maps and bad pixel masks are
taken through the equivalent reduction process as the data, providing
reliable and accurate errors and pixel quality in the final
one-dimensional spectrum.  On a $1.7$~GHz Pentium~IV processor {\tt
iSPEC} can reduce $60$ nights of data in $\sim3.75$ hours, or $\sim4$
minutes per $30-40$ spectra.  The software includes all the required
two-dimensional data reduction algorithms, including an interactive
one-dimensional spectrum extraction routine, and a suite of
one-dimensional spectrum analysis routines.  [{\bf Cite Andy and
Karl's use of iSPEC?}]  The {\tt iSPEC} software and documentation is
available for public consumption at {\tt
http:cerebus.as.arizona.edu/\~{}ioannis/ispec.html}.

This data reduction package was written for the B\&C Spectrograph at
the 2.3-m telescope at Kitt Peak National Observatory.  However, it
has been written with generality in mind and could be applied to other
longslit spectroscopy data after some modifications.  Although some of
the routines are in the spirit of the IRAF \emph{twodspec} and
\emph{onedspec} packages, this data reduction software is much more
automated, and much less of a black box.  Nevertheless, the software
was written to be fully compatible with IRAF.

\section{Initial Reductions}

Overscan subtraction, trimming, flat-fielding, initializing the
error maps, and determining the wavelength solution for the data
frames all fall under the general category of {\em initial
reductions}.  After initial user input, these procedures are largely
automated and are not time consuming.

\subsection{Preliminary CCD Processing}\label{sec:preproc}

The noise properties of the CCD must be known to ensure correct
propegation of errors through the data reduction process, and to
ensure proper weighting of bad pixels.  To this end, the gain and the
read noise should be verified using a pair of dome flats and bias
frames with the routine {\tt GAIN\_RDNOISE}.  The gain $\Gamma$ in
$e^{-}\ {\rm ADU}^{-1}$ (analog-to-digital units) and the read noise
$r_{e}$ in $e^{-}$ are given by

\begin{equation}
\Gamma = \frac{(\bar{F_{1}}+\bar{F_{2}}) -
(\bar{B_{1}}+\bar{B_{2}})} {\sigma^{2}_{F_{1}-F_{2}} -
\sigma^{2}_{B_{1}-B_{2}}},
\end{equation}

\begin{equation}
r_{e} = \frac{\Gamma \ \sigma_{B_{1}-B_{2}}}{\sqrt{2}}.
\end{equation}

\noindent In these equations the mean and dispersion of the dome flat
and the bias frame are given by $\bar{F}$ and $\sigma_{F}$, and
$\bar{B}$ and $\sigma_{B}$, respectively.  These quantities are
measured \emph{after} subtraction of the overscan region which is not
subject to Poisson statistics.  Also note that if $N$ bias frames were
combined to make a master bias frame, the derived read noise must be
multiplied by $\sqrt{N}$. 
%The gain should not differ by more than a few percent (it is a
%property of the CCD detector), while the read noise will differ from
%the lab measurement by $\pm1$~e$^{-}$ due to environmental conditions
%at the telescope at the time of the observations.
The image headers should be updated with these measured quantities,
which will be used to initialize the error maps (\S~\ref{sec:proc}).

At this point a few cross-sections of images should be examined to
determine or verify the good data sections (how to trim the images)
and the overscan region.  Bad pixels, columns, and cosmic rays are
kept in the images.  The philosophy adopted was that \emph{ignoring}
bad data introduces fewer systematics than attempting to \emph{repair}
bad data.  (These bad pixels can be interpolated over at the end for
aesthetic purposes, but are not used for measurements.)  Stacks of
bias frames or dome flats can be combined into master frames with
iterative sigma clipping using {\tt RKCOMBINE}.

\subsection{CCD processing}\label{sec:proc}

The driver routine for processing the CCD images is {\tt RKCCDPROC}.
These processing steps include overscan subtraction, trimming, bias
subtraction, and flat fielding.  The master flat field is created
using a dome flat and a sky flat; this procedure is described in
\S~\ref{sec:flat}.  The user has the option of processing an entire
directory of images, or a specified FITS file list.  The user can
specify the dimensions of the overscan region and the data section.
The default overscan region is taken from the header, less two columns
on either side to avoid edge effects.  The overscan section is
collapsed to its median along the dispersion direction.  A low-order
polynomial (default second order) is fitted to this median overscan
region and subtracted from all the data columns.  The two dimensional
images are subsequently trimmed.

The noise properties of the CCD and the registered counts in each
pixel are used to initialize variance maps for each image.  According
to Poisson statistics, the variance in a pixel $i$ registering $f_{i}$
ADU is given by

\begin{equation}
\sigma_{i}^{2} = \left[ \frac{f_{i}}{\Gamma} + \left(\frac{r_{\rm
e}}{\Gamma}\right)^{2} \right],
\end{equation}

\noindent where $\Gamma$ and $r_{e}$ were defined in
\S~\ref{sec:preproc}.  The variance maps characterize the
signal-to-noise of the observed galaxy spectrum.  In addition, the
variance of cosmic-ray contaminated pixels or bad columns can be set
to a very large value, effectively discarding that pixel when doing
science extractions.

\subsubsection{Flat-fielding and illumination correction}\label{sec:flat}

The pixel-to-pixel quantum efficiency variations of the CCD are
modeled with a dome flat exposure, while the illumination function
(gradients in the spatial illumination) is determined from a sky flat.
The procedure that generates the master flat field is called {\tt
RKMASTERFLAT}.  The procedure to generate the master flat field is
straightforward.  A high-order b-spline is fitted to the median
average of several rows centered around the middle row of the CCD.
This fit should achieve $1\%$ residuals or better.  Every row of the
dome flat is divided by this fit.  The sky flat is then divided by
this preliminary flat field.  The most troublesome aspect of twilight
flats is that they contain many spectral lines (from the Sun).
Consequently, the sky flat is smoothed by median-averaging several
columns.  A low-order polynomial is then fitted at every row along the
column dimension.  This procedure generates the illumination
function.  The sky flat is normalized to a mean of one and multiplied
by the preliminary flat, generating the master flat field.

% Slit function.

\subsection{Wavelength calibration}\label{sec:wave}

The He-Ar calibration lamp exposed closest to the zenith is chosen to
provide the wavelength solution.  Unlike IRAF, this step is fully
automated and results in a full two-dimensional map between pixel
position and wavelength.  The only inputs are initial guesses for the
spectral range and the dispersion, which are known from the grating
setting.  

%The traditional way to determine the wavelength solution (e.g. using
%IRAF) is by manually clicking on arc lines in a spatial cross-section
%of the calibration lamp to obtain line centroids in pixels, and typing
%in the corresponding wavelength in Angstroms.  IRAF routines such as
%IDENTIFY and REIDENTIFY generate two-dimensional wavelength solutions,
%once an initial solution has been found.  

We begin by searching for He-Ar emission lines at the center row of
the arc lamp image, and tracing those centroids up and down in the
spatial dimension.  This traceset is used as an initial guess to trace
the arc lines starting at the first (zeroth) row, upwards.  The
traceset is trimmed to exclude lines that are blends, or that deviate
from their median pixel position by more than $\sim2$ pixels.  Lines
with low signal-to-noise [S/N] are also removed.  The final traceset
contains $50-100$ lines.  A low-order function (polynomial, legendre,
or chebyshev) is fit to each traced line \emph{individually} such that
ultimately the distortions in the chip are summarized by an
$[{\mathbf{ncoeff}},{\mathbf{nlines}}]$ matrix, where
${\mathbf{ncoeff}}$ is the number of coefficients, and
${\mathbf{nlines}}$ is the number of traced arc lines.  Note that the
variance of each pixel is incorporated into the fits, so that not only
do we obtain errors in the line centroids, but bad pixels due to
cosmic rays or bad columns do not affect the fitting.

The wavelength solution for each image is individually adjusted using
a list of well-known terrestrial sky lines.  A table of these sky
lines, their transitions, and wavelengths can be found below.

\section{Science Reductions}

{\em Science reductions} include flux calibration and extracting
sky-subtracted one-dimensional spectra.  Although these procedures can
also be automated, in general they demand more user interaction.

\subsection{Flux calibration}

Flux calibration involves generating a sensitivity curve from standard
star observations that converts counts into physical units (${\rm erg\
s}^{-1}\ {\rm cm}^{-2}\ {\rm \AA}^{-1}$).  Describe how the object and
sky aperture windows are defined.

The extraction aperture should be chosen to be large enough so that
all the light is summed.  Once the standard star spectrum is summed,
we can generate a sensitivity curve.

\subsection{Determining the sky spectrum}

Once the wavelength solution is known at all spatial positions on the
CCD we can estimate the sky spectrum.  The following procedure
generates a ``supersky'' spectrum by sampling the sky at many more
wavelengths than the object spectra are sampled.  The user should
choose as many rows that are uncontaminated by the object or standard
star to generate the supersky spectrum.  For example, imagine the CCD
has $1200$ pixels in the wavelength direction and the upper and lower
$40$ rows are chosen as the sky rows.  The supersky spectrum will
therefore be sampled at $1200\times2\times40=96\,000$ points.  The sky
is iteratively fit in a least-squares sense with a low-order (by
default 4) b-spline.  This fit is very robust against cosmic rays or
bad pixels, and behaves well at the endpoints by virtue of the choice
of breakpoints.  Instead of interpolating this highly sampled spectrum
to the poorer resolution of the image rows (introducing unnecessary
interpolation errors), the parameters of the fit are stored and
evaluated at all the spatial rows using the two-dimensional wavelength
solution.

\subsection{Additional utilities}

Conservatively rejecting cosmic rays, cleaning dead CCD columns, and
replacing masked pixels with interpolated values or with local
medians.

Measuring the gain and read noise using a pair of bias frames and dome
flats.  

\acknowledgments

\begin{references}{}

\reference{} Green, R. et al. 1994, Steward Observatory 2.3-m Boller
and Chivens Spectrograph Manual

\reference{} Massey, P. 1997, A User's Guide to CCD Reductions with
IRAF

\reference{} Massey, P., Valdes, F., \& Barnes, J. 1992, A User's
Guide to Reducing Slit Spectra with IRAF

\end{references}

\clearpage

\section*{{\tt RKSPEC} Recipe}

This recipe describes how to reduce an entire run (consisting of any
number of nights of data) of B\&C longslit spectroscopy observations.
Make a list of the raw fits files for the run.  For example, in IRAF
type

\begin{quote}
cl$>$ imhead *.fits $>$ list \\
cl$>$ !\,lpr list
\end{quote}

\noindent I highlight the calibration frames (bias frames, dome flats,
and comparison arc lamps) and standard star observations different
colors so I can identify them with a glance.  The batch-processing
capabilities of {\tt RKSPEC} revolve around text file lists of objects
to process.  Preliminary steps are to generate a master bias frame and
a master flat field.  If multiple bias frames and dome flats were
taken then combine them with {\tt RKCOMBINE} using minmax or sigma
clipping to reject outliers.  Wavelength calibration requires HeAr arc
lamp calibration frames.  Sky flat (twilight sky) observations are
also required to determine the illumination function in the flat
field.  Finally, examine the FITS headers for keywords such as
EXPTIME, OBJECT, DATE-OBS etc.  The effective airmass and parallactic
angle are computed and inputed into the headers during data
processing.

Examine the bias frames from several of the nights to determine the
overscan region and the trim section.  Write these numbers down with
respect to pixel $(0,0)$.  I use the dome flat to determine the
overscan region, which should be several columns away from the last
good data pixel (the edge of the trim region).  Select the master bias
frame, dome flat, and sky flat and input them in the parameter file
(below) along with the overscan and trim regions, zero-indexed.  Look
up the gain and the read noise and input them in the parameter file.

The first text file to create for batch processing the data is the
parameter file.  Create one parameter file for each night that must be
reduced.  My naming convention is rkbatch\_01nov10.txt.

\begin{center}

{\scriptsize
\parbox{7in}{

\setlength{\leftmargin}{0.5in}

\# input text file for RKBATCH

bias.fits	  	\# master bias frame                                              \\
dflat.fits	  	\# master dome flat						  \\
sflat.fits	  	\# sky flat							  \\
arclamp.fits	  	\# arc lamp							  \\
masterflat.fits 	\# input/output flat field name					  \\
wmap.idlsave		\# input/output wavelength map name				  \\
sensfunc.fits		\# input/output sensitivity function name			  \\
badpix.dat	  	\# input bad pixel file						  \\
1.5		  	\# gain (electron/ADU)						  \\
5.8		  	\# read noise (electron)					  \\
10 1194 0 125	  	\# trim region (zero-indexed)					  \\
1202 1217 0 125		\# overscan region (zero-indexed)				  \\
50		  	\# b-spline order for the flat fit				  \\
25			\# b-spline order for the sensitivity function fit		  \\
3620.0		  	\# output minimum wavelength (and starting wavelength guess)	  \\
2.75		  	\# output wavelength dispersion (and starting dispersion guess)	  \\
40		  	\# standard star spectrum extraction aperture (arcsec)		  \\
20		  	\# standard star spectrum sky aperture (arcsec)			  \\
4.5		  	\# cosmic ray clipping threshold (sigma, LA\_COSMIC)		  \\
4.0		  	\# object clipping threshold (sigma, LA\_COSMIC)		  \\
0.1		  	\# sigma fraction (LA\_COSMIC)                                    \\

}
}

\end{center}

\noindent Every parameter is explained in the comments above.  

Next either create or obtain a bad pixel mask.  The standard stars are
good spectra with which to identify bad columns while the dome flats
are good indicators of dead pixels.  For compatibility with IRAF, the
bad pixel file should contain row and column numbers with respect to
pixel $(1,1)$.  

First make the flat field to ensure the order of the dome flat fit is
good and that the sky flat is flat!  This procedure will also enable
you to determine if the trim region was selected correctly.

\begin{quote}
IDL$>$ rkbatch, paramfile, /makeflat
\end{quote}

\noindent Next compute the wavelength solution to determine the
starting wavelength.  You should have a good guess of the starting
wavelength in the parameter file otherwise the cross-correlation
technique will not work.  Process the arc lamp and determine the
wavelength map.

\begin{quote}
IDL$>$ rkbatch, paramfile, proclist='a.3447.fits', /ccdproc, /arcfit
\end{quote}

\noindent Update the parameter file with the desired output starting
wavelength and dispersion.  Look at the overscan-subtracted bias frame
to make sure the overscan correction was done well.  Do all of the
above shit for all the nights, making a parameter file for each
night.  It is recommended that the starting wavelength and dispersion,
trim and overscan regions are all defined the same way.

Now it is time to make a whole shitload of lists.  (1) proclist; (2)
crsplitfile; (3) crfile; (4) stdfile; (5) tracefile; (6) calibfile.
Include examples of all the text files!

\begin{enumerate}

\item[(1)]{proclist: list of files to process (flat-field).  This list
should include all objects, standard stars, sky flats and arc lamps,
but should exclude the dome flats and the bias frames.  The latter
frames are specified in the parameter file.}

\item[(2)]{crsplitfile (optional): If cosmic ray-split images were
obtained these images must be listed in this file.  Each line in the
text file should contain the images that must be combined.  For
example, if three images were obtained of one object, a.340[5-7].fits
then one line of this file would look like

\begin{quote}
ra.3405.fits ra.3406.fits ra.3407.fits
\end{quote}

If this file does not exist then this processing step is ignored
quietly.  Note the ra....}

\item[(3)]{crfile: list of objects in which to look for cosmic rays.
Basically proclist but without the the arclamps and with cosmic ray
combined image filenames combined.  For example, above the new file
name would be ra.3405\_3.fits to replace the individual file names
given in proclist.  Also take out sky flats.}

\item[(4)]{stdfile:  list of standards to use to derive the
sensitivity function.}

\item[(5)]{tracefile:  Images with which to model the tilt in the
spectral light, normally standard stars.}  

\item[(6)]{calibfile:  Images to wavelength- and flux-calibrate.
Basically the crfile but with a c prepended.}

\end{enumerate}

Make the rest of the parameter files.  And make all these files for
the remaining nights.  Should take about an hour if meticulous.

Preliminary steps are to go through and make the flat fields for all
the nights and determine the wavelength solution so that the parameter
files can be updated accordingly with the starting wavelength, which
should be larger than the largest starting wavelength of the arc lamp
solutions. 




\end{document}