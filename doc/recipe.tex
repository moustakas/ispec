\documentclass[preprint,11pt]{aastex}

\begin{document}

%\newcommand{\begv}{\begin{verbatim}}
%\newcommand{\endv}{\end{verbatim}}

\input{/home/ioannis/bin/latex/myprefs}

\title{RKSPEC Recipe}
\author{J. Moustakas \\ 2002 September}

\section{Preliminaries}

[NOTE:  The bad pixel file input needs to be in the coordinates of the
untrimmed image, and one-indexed (not zero-indexed).]



The purpose of this document is to describe how to reduce one whole
night of longslit spectroscopy in semi-batch mode using RKSPEC.  Once
the input parameter files have been set up and the appropriate
calibration images have been chosen, the data can be reduced very
quickly.  The end of the manual describes how to reduce data from an
entire observing run.  Throughout the manual I will attempt to
indicate parts of the code that have not yet been generalized.

At minimum the data headers must contain the following keywords:
OBJECT, IMAGETYP, DATE\_OBS, EXPTIME, RA, DEC, OBSERVAT,
GAIN, and RDNOISE.  Note the spelling of the keywords.  The DATE\_OBS
header keyword must be in FITS standard format (YYYY-MM-DD).

The driver routine for batch-reducing one night of data is called
{\tt RKBATCH}, which requires an input parameter file, an example of
which follows.  This parameter file will be called rkbatch.txt.

\begin{center}
\begin{verbatim}
# input text file for RKBATCH

bias.fits                 # master bias frame
domeflat.fits             # master dome flat
skyflat.fits              # sky flat
rarclamp.fits             # arc lamp
masterflat.fits           # input/output flat field name
wmap.idlsave              # input/output wavelength map name
trace.idlsave             # input/output trace structure
sens.fits                 # input/output sensitivity function name
badpix.dat                # input bad pixel file
1.5                       # gain (electron/ADU)
5.8                       # read noise (electron)
3 1194 0 125              # trim region (zero-indexed)
1202 1217 0 125           # overscan region (zero-indexed)
50                        # b-spline order to the flat fit
20                        # b-spline order to the sensitivity function fit
3605.0                    # output minimum wavelength (and starting wavelength guess)
2.75                      # output wavelength dispersion (and starting dispersion guess)
30                        # standard star spectrum extraction aperture (arcsec)
60                        # standard star spectrum sky aperture (arcsec)
4.5                       # cosmic ray clipping threshold (sigma, LA_COSMIC)
2.0                       # object clipping threshold (sigma, LA_COSMIC)
0.5                       # sigma fraction (LA_COSMIC)
\end{verbatim}
\end{center}

\noindent Before beginning the reductions, the user must (1) select
which images (file names) to use as the bias frame, the dome flat, the
arc lamp, and (optionally) the sky flat; (2) choose names for the
master flat field, wavelength map, and spatial distortion map (called
the {\em trace structure} in the above example; (3) optionally create
an input bad pixel mask with known CCD bad pixels and dead columns;
(4) input the CCD gain and read noise; (5) determine the trim region
(or the usable data section) and the overscan region.  The approximate
dispersion (\AA~pixel$^{-1}$) and starting wavelength should be
specified in the parameter file as initial guesses to the wavelength
solution.  The last three parameters in the parameter file refer to
cosmic-ray rejection with LA\_COSMIC (van Dokkum, P. 2001, PASP, 113,
1420).  The defaults are conservative and relatively robust, but I
recommend that the individual pixel masks are examined to ensure that
no important spectral features have been affected by the cosmic-ray
rejection.

In addition to the parameter file the user will need to create simple
text files containing the lists of files to process, cosmic-ray clean,
flux-calibrate, etc.  

\section{Running RKBATCH}

The routine RKBATCH carries out eight distinct tasks, as specified by
the following IDL keywords: MAKEFLAT, CCDPROC, DISTORTION, ARCFIT,
CRSPLITS, CRCLEAN, MAKESENS, and CALIBRATE.  These keywords are
described in greater detail in the sub-sections that follow.

\subsection{MAKEFLAT}

The keyword MAKEFLAT generates a master flat-field from the dome flat.
Optionally, a sky flat can be used to determine the illumination
correction along the spatial dimension of the CCD.  The flat field is
created by the routine RKMASTERFLAT, which is called by RKCCDPROC.
The name of the output flat field is specified in the input parameter
file.

\begin{verbatim}
IDL> 

\end{verbatim}

\subsection{CCDPROC}

This keyword 

, the
routine that is responsible for trimming, overscan-subtraction,
updating FITS headers, initializing the variance maps and bad pixel
masks, bias-subtraction, and flat-fielding.  



\subsection{DISTORTION}



\subsection{ARCFIT}

\subsection{CRSPLITS}

\subsection{CRCLEAN}

\subsection{MAKESENS}

\subsection{CALIBRATE}

\section{Error propagation}

The noise properties of the CCD and the registered counts in each
pixel are used to initialize variance maps for each image.  According
to Poisson statistics, the variance in a pixel $i$ registering $f_{i}$
ADU is given by

\begin{equation}
\sigma_{i}^{2} = \left[ \frac{f_{i}}{\Gamma} + \left(\frac{r_{\rm
e}}{\Gamma}\right)^{2} \right],
\end{equation}

\noindent where $\Gamma$ is the gain in $e^{-}\ {\rm ADU}^{-1}$
(analog-to-digital units) and the read noise is $r_{e}$ in $e^{-}$.  


\end{document}